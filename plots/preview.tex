\documentclass[fleqn]{article}
\pagenumbering{arabic}

%\usepackage{showkeys}
\usepackage{url}
\usepackage{graphicx}
\usepackage{float}
\usepackage{caption}
\usepackage{subcaption}
\usepackage{fullpage}
%\usepackage{html}
\usepackage[T1]{fontenc}
\usepackage [latin2]{inputenc}
\usepackage {times}
\usepackage{dsfont}
%\usepackage[english]{babel}
\usepackage {amsmath}
\usepackage {amsthm}
\usepackage {amssymb}
\usepackage{calc}
\usepackage {bm}


\usepackage{tikz}
\usepackage[percent]{overpic}

% for tikz matlab plots
\usepackage{pgfplots}
\pgfplotsset{compat=newest}

% ./Figures added to the environmental variable TEXINPUT
% import image path under current directory

% import path of images not under the current directory
%\usepackage{import}
%\import{<path>/<to>/<file>}{<filename>.tex}
%\graphicspath{
%			{./webnode/},
%		 	}

\newenvironment{source}
{\begin{list}{}{\setlength{\leftmargin}{1em}}\item\scriptsize\bfseries}
	{\end{list}}

\newenvironment{tinysource}
{\begin{list}{}{\setlength{\leftmargin}{1em}}\item\tiny\bfseries}
	{\end{list}}


\title{PAM}
\author{Lourens Touwen}


\begin{document}
	
	\pagestyle{plain}
	
	\maketitle
	
	\newlength\fheight 
	\newlength\fwidth 
	\setlength\fheight{0.44\textwidth} 
	\setlength\fwidth{0.65\textwidth}
	
	\definecolor{mycolor1}{rgb}{0.00000,0.80000,0.80000}
	\definecolor{mycolor2}{rgb}{0.00000,0.50000,0.50000}
	\definecolor{mycolor3}{rgb}{0.00000,0.30000,0.30000}
	\definecolor{mycolor4}{rgb}{0.00000,0.10000,0.10000}
	\definecolor{mycolor5}{rgb}{0.00000,1,1}
	
	Here we generated Preferential Attachment Models (PAMs) as defined in chapter 8 of RGCN volume 1. 
	
	From 8.4.11 en 8.4.12, we know degree distribution behaves like powerlaw with exponent $\tau = 3 + \frac{\delta}{m} > 2$.
	
	The objective here is to inspect three different definitions for a degree distribution. The conventional definition, the size-biased definition (1.2.2 in RGCN I), and the random friend distribution (the degree distribution of a uniformly selected adjacent vertex to a uniformly selected vertex in the network).
	
	The plots below are for two generated graphs. One with parameters $\delta= -1, m = 2 \implies \tau = 2.5$, the other with parameters $\delta = 1, m = 2 \implies \tau = 3.5$. Both with $n = 10000$. 
	
	\begin{figure}[!htb]
		\centering
		%Degree distributions of PAM (t=2000, delta=-1, m=2), follow powerlaw (tau=2.5) by RGCN volume 2, 8.4.11 en 8.4.12. 
\tikzstyle{every node}=[font=\normalsize]

\begin{tikzpicture}
    \begin{axis}[%
width=0.951\fwidth,
height=\fheight,
at={(0\fwidth,0\fheight)},
scale only axis,
xminorticks=true,
yminorticks=true,
axis x line*=bottom,
axis y line*=left,
x tick label style = {/pgf/number format/.cd, scaled x ticks = false, set thousands separator={}, fixed},
axis background/.style={fill=white},
legend style={legend pos=north east, legend cell align=left, align=left, draw=white!15!black}, 
xmin=0, 
xmax=418, 
ymin=0, 
ymax=1, 
xlabel={Degrees}, 
ylabel={$P(X > x)$}]

\addplot [color=mycolor2, line width = 2pt, mark size=1.5pt, mark=*, only marks, mark options={solid, mycolor2}]
table[row sep=crcr]{
1	0.9905\\ 
2	0.3945000000000103\\ 
3	0.23650000000001017\\ 
4	0.15500000000001013\\ 
5	0.1145000000000101\\ 
6	0.09350000000001008\\ 
7	0.07400000000001006\\ 
8	0.061000000000010046\\ 
9	0.05300000000001004\\ 
10	0.04500000000001003\\ 
11	0.03950000000001003\\ 
12	0.03250000000001002\\ 
13	0.02900000000001002\\ 
14	0.02550000000001002\\ 
15	0.022500000000010022\\ 
16	0.02100000000001002\\ 
17	0.019000000000010023\\ 
18	0.018500000000010022\\ 
19	0.01700000000001002\\ 
20	0.015000000000010021\\ 
21	0.013500000000010021\\ 
22	0.012000000000010022\\ 
23	0.011500000000010021\\ 
24	0.010500000000010022\\ 
28	0.009500000000010021\\ 
29	0.00900000000001002\\ 
31	0.007000000000010021\\ 
32	0.006000000000010021\\ 
33	0.005500000000010021\\ 
34	0.005000000000010021\\ 
40	0.00450000000001002\\ 
44	0.00400000000001002\\ 
45	0.00350000000001002\\ 
46	0.00300000000001002\\ 
48	0.00250000000001002\\ 
60	0.00200000000001002\\ 
64	0.0015000000000100198\\ 
68	0.0010000000000100198\\ 
237	0.0005000000000100198\\ 
418	1.0019762797242038e-14\\ 
};
\addlegendentry{Degree distribution}

\addplot [color=mycolor1, line width = 2pt, mark size=2pt, mark=diamond*, only marks, mark options={solid, mycolor1}]
table[row sep=crcr]{
1	0.9975931086901444\\ 
2	0.6955915885482682\\ 
3	0.575500380035473\\ 
4	0.4929060045604282\\ 
5	0.4416012161135062\\ 
6	0.4096782366354214\\ 
7	0.3750950088674962\\ 
8	0.348745882949077\\ 
9	0.3305041803901714\\ 
10	0.3102356219913874\\ 
11	0.29490752470230697\\ 
12	0.2736255383835838\\ 
13	0.2620977957942754\\ 
14	0.24968330377502018\\ 
15	0.2382822396757042\\ 
16	0.232201672156069\\ 
17	0.2235875348365858\\ 
18	0.2213073220167226\\ 
19	0.2140866480871558\\ 
20	0.2039523688877638\\ 
21	0.1959716240182426\\ 
22	0.18761084367874423\\ 
23	0.18469723840891902\\ 
24	0.17861667088928382\\ 
28	0.17152267544970942\\ 
29	0.16784899923992982\\ 
31	0.15214086648087224\\ 
32	0.14403344312135866\\ 
33	0.13985305295160946\\ 
34	0.13554598429186787\\ 
40	0.13047884469217189\\ 
44	0.12490499113250629\\ 
45	0.1192044590828483\\ 
46	0.1133772485431979\\ 
48	0.10729668102356271\\ 
60	0.09969597162401872\\ 
64	0.09158854826450513\\ 
68	0.08297441094502193\\ 
237	0.05295160881682315\\ 
418	-2.0816681711721685e-17\\ 
};
\addlegendentry{Size biased degree distribution}

\addplot [color=mycolor3, line width = 2pt, mark size=1.5pt, mark=square*, only marks, mark options={solid, mycolor3}]
table[row sep=crcr]{
1	0.9998771898084249\\ 
2	0.8545326347387832\\ 
3	0.7498000904001487\\ 
4	0.6616109122233906\\ 
5	0.6015767553076492\\ 
6	0.5637036712524682\\ 
7	0.520916482718662\\ 
8	0.4868988112619893\\ 
9	0.4641867949424935\\ 
10	0.43818741727764043\\ 
11	0.4177078519273349\\ 
12	0.38953763547972414\\ 
13	0.37306732806041065\\ 
14	0.35705532979198706\\ 
15	0.340723918634006\\ 
16	0.33322906665701707\\ 
17	0.320232977001059\\ 
18	0.3176469464602802\\ 
19	0.3081994093912261\\ 
20	0.2932529607457513\\ 
21	0.2819082297480882\\ 
22	0.2701365355211648\\ 
23	0.2658806379377356\\ 
24	0.25702219314879016\\ 
28	0.24790807641756815\\ 
29	0.2433646003030394\\ 
31	0.2201111553477384\\ 
32	0.2076206684501146\\ 
33	0.20166004767469117\\ 
34	0.1957771776207685\\ 
40	0.18903672139422797\\ 
44	0.18088970330814738\\ 
45	0.17367115554523915\\ 
46	0.1644612506931468\\ 
48	0.156439742020898\\ 
60	0.14547414289049623\\ 
64	0.1358559263370002\\ 
68	0.12256131480817814\\ 
237	0.07863599053022251\\ 
418	0.00011627553732819484\\ 
};
\addlegendentry{Random friend degree distribution}

	\end{axis}
\end{tikzpicture}

		\caption{Degree distribution tau2,5}
	\end{figure}
	
	\begin{figure}[!htb]
		\centering
		%Loglog Degree distributions of PAM (t=10000, delta=-1, m=2), follow powerlaw (tau=2.5) by RGCN volume 2, 8.4.11 en 8.4.12. 
\tikzstyle{every node}=[font=\normalsize]

\begin{tikzpicture}
    \begin{axis}[%
width=0.951\fwidth,
height=\fheight,
at={(0\fwidth,0\fheight)},
scale only axis,
xminorticks=true,
yminorticks=true,
axis x line*=bottom,
axis y line*=left,
x tick label style = {/pgf/number format/.cd, scaled x ticks = false, set thousands separator={}, fixed},
axis background/.style={fill=white},
legend style={legend pos=south west, legend cell align=left, align=left, draw=white!15!black}, 
xmin=0, 
xmax=701, 
ymin=0, 
ymax=1, 
xmode=log, 
ymode=log, 
xlabel={Degrees}, 
ylabel={$P(X > x)$}]

\addplot [color=mycolor2, line width = 2pt, mark size=1.5pt, mark=*, only marks, mark options={solid, mycolor2}]
table[row sep=crcr]{
1	0.9978\\ 
2	0.40080000000004945\\ 
3	0.22790000000005217\\ 
4	0.15170000000005102\\ 
5	0.1082000000000508\\ 
6	0.08430000000005089\\ 
7	0.06880000000005093\\ 
8	0.05640000000005095\\ 
9	0.048400000000050944\\ 
10	0.04140000000005094\\ 
11	0.03590000000005093\\ 
12	0.032300000000050934\\ 
13	0.028800000000050938\\ 
14	0.02620000000005094\\ 
15	0.02400000000005094\\ 
16	0.02070000000005094\\ 
17	0.018800000000050943\\ 
18	0.01730000000005094\\ 
19	0.016500000000050943\\ 
20	0.015100000000050942\\ 
21	0.014000000000050942\\ 
22	0.013300000000050941\\ 
23	0.01230000000005094\\ 
24	0.01130000000005094\\ 
25	0.010800000000050939\\ 
26	0.010000000000050939\\ 
27	0.009300000000050938\\ 
28	0.008600000000050938\\ 
29	0.008200000000050939\\ 
30	0.00810000000005094\\ 
31	0.00790000000005094\\ 
32	0.007800000000050939\\ 
33	0.007400000000050939\\ 
34	0.006700000000050939\\ 
35	0.006500000000050939\\ 
36	0.006200000000050939\\ 
37	0.006000000000050939\\ 
39	0.00550000000005094\\ 
40	0.00530000000005094\\ 
41	0.0051000000000509405\\ 
42	0.004900000000050941\\ 
43	0.004800000000050941\\ 
44	0.004500000000050941\\ 
46	0.00440000000005094\\ 
47	0.00430000000005094\\ 
49	0.00390000000005094\\ 
50	0.00360000000005094\\ 
51	0.00340000000005094\\ 
52	0.00330000000005094\\ 
53	0.00310000000005094\\ 
61	0.0030000000000509402\\ 
62	0.0029000000000509404\\ 
63	0.0028000000000509406\\ 
64	0.0026000000000509405\\ 
65	0.0025000000000509406\\ 
70	0.002400000000050941\\ 
76	0.002300000000050941\\ 
79	0.002200000000050941\\ 
80	0.0021000000000509413\\ 
83	0.0019000000000509412\\ 
84	0.0018000000000509412\\ 
85	0.0017000000000509411\\ 
86	0.001600000000050941\\ 
94	0.001500000000050941\\ 
103	0.001400000000050941\\ 
119	0.001300000000050941\\ 
133	0.001200000000050941\\ 
134	0.0011000000000509409\\ 
148	0.0010000000000509408\\ 
153	0.0009000000000509408\\ 
165	0.0008000000000509407\\ 
182	0.0007000000000509407\\ 
209	0.0006000000000509406\\ 
233	0.0005000000000509406\\ 
245	0.0004000000000509406\\ 
248	0.0003000000000509406\\ 
299	0.00020000000005094063\\ 
701	0.00010000000005094062\\ 
};
\addlegendentry{Degree distribution}

\addplot [color=mycolor1, line width = 2pt, mark size=2pt, mark=diamond*, only marks, mark options={solid, mycolor1}]
table[row sep=crcr]{
1	0.9994482343499197\\ 
2	0.6999899678972885\\ 
3	0.5698986757624558\\ 
4	0.4934540529695153\\ 
5	0.43890449438203366\\ 
6	0.4029394060995289\\ 
7	0.3757273274478429\\ 
8	0.35084771268058706\\ 
9	0.3327899277688691\\ 
10	0.3152337479935878\\ 
11	0.30006019261638034\\ 
12	0.28922552166934956\\ 
13	0.2778140048154167\\ 
14	0.2686847913322704\\ 
15	0.26040830658106634\\ 
16	0.24716593097913986\\ 
17	0.2390650080256886\\ 
18	0.23229333868379437\\ 
19	0.22848113964687614\\ 
20	0.2214586677367636\\ 
21	0.21566512841092075\\ 
22	0.21180276886035887\\ 
23	0.20603430979133786\\ 
24	0.20001504815409854\\ 
25	0.19688001605136973\\ 
26	0.191663322632429\\ 
27	0.18692315409310303\\ 
28	0.18200742375602424\\ 
29	0.1790981139646919\\ 
30	0.17834570626003698\\ 
31	0.1767907303370835\\ 
32	0.17598816211878493\\ 
33	0.1726775682183033\\ 
34	0.16670846709470763\\ 
35	0.1649528491171795\\ 
36	0.16224418138042182\\ 
37	0.16038824237560637\\ 
39	0.15549759229534943\\ 
40	0.153491171749603\\ 
41	0.1514345906902129\\ 
42	0.14932784911717914\\ 
43	0.14824939807384044\\ 
44	0.1449388041733588\\ 
46	0.1437851123595546\\ 
47	0.14260634028892857\\ 
49	0.13769060995184979\\ 
50	0.1339285714285752\\ 
51	0.1313703852327485\\ 
52	0.13006621187801332\\ 
53	0.12740770465489928\\ 
61	0.12587780898876763\\ 
62	0.12432283306581414\\ 
63	0.12274277688603882\\ 
64	0.11953250401284451\\ 
65	0.11790228731942554\\ 
70	0.11614666934189741\\ 
76	0.11424056982343829\\ 
79	0.11225922953451369\\ 
80	0.11025280898876724\\ 
83	0.10608948635634338\\ 
84	0.10398274478330963\\ 
85	0.10185092295345403\\ 
86	0.09969402086677662\\ 
94	0.09733647672552455\\ 
103	0.094753210272876\\ 
119	0.09176865971107817\\ 
133	0.08843298555377471\\ 
134	0.08507223113964943\\ 
148	0.08136035313001852\\ 
153	0.07752307383627846\\ 
165	0.07338483146067643\\ 
182	0.06882022471910328\\ 
209	0.0635784510433407\\ 
233	0.0577347512038542\\ 
245	0.051590088282505726\\ 
248	0.04537018459069177\\ 
299	0.03787118780096445\\ 
701	0.020289927768861273\\ 
};
\addlegendentry{Size biased degree distribution}

\addplot [color=mycolor3, line width = 2pt, mark size=1.5pt, mark=square*, only marks, mark options={solid, mycolor3}]
table[row sep=crcr]{
1	0.9999622452472311\\ 
2	0.8576148394796365\\ 
3	0.7438619833295332\\ 
4	0.6635191735615715\\ 
5	0.6007737291551947\\ 
6	0.5573124617748997\\ 
7	0.5229273544479384\\ 
8	0.4903609480592135\\ 
9	0.46638311642615654\\ 
10	0.4429208410616105\\ 
11	0.4224535661017733\\ 
12	0.407689140502085\\ 
13	0.3919813089589862\\ 
14	0.3789166241753616\\ 
15	0.3672288863201298\\ 
16	0.3491419324358425\\ 
17	0.3377119320523928\\ 
18	0.3283835123178632\\ 
19	0.32310296571807584\\ 
20	0.31272237499994254\\ 
21	0.3049374311561134\\ 
22	0.2993876990497396\\ 
23	0.29125255239283687\\ 
24	0.2828475163116223\\ 
25	0.2783722324544067\\ 
26	0.27086610712499054\\ 
27	0.26442078884700665\\ 
28	0.25786144614886713\\ 
29	0.25372878871606114\\ 
30	0.2526356077124769\\ 
31	0.2504141589796512\\ 
32	0.24927057922607143\\ 
33	0.24463334600455358\\ 
34	0.23595764631447858\\ 
35	0.23342211622410014\\ 
36	0.22993863521031385\\ 
37	0.22740108144985732\\ 
39	0.2210186229279028\\ 
40	0.21819427867947802\\ 
41	0.21546221436865934\\ 
42	0.21239199399394168\\ 
43	0.2111014651273834\\ 
44	0.2064681069075912\\ 
46	0.20456713753879574\\ 
47	0.20316030929131568\\ 
49	0.1964672400905932\\ 
50	0.19081770851751087\\ 
51	0.1873339174710976\\ 
52	0.1855073652902747\\ 
53	0.18198533071909967\\ 
61	0.17975467979471835\\ 
62	0.17767121990335524\\ 
63	0.1754003031938074\\ 
64	0.17089392358043498\\ 
65	0.168456647251265\\ 
70	0.16602601538784734\\ 
76	0.1634051192080038\\ 
79	0.1604651592253369\\ 
80	0.15760916414151574\\ 
83	0.15182044373221018\\ 
84	0.14885330486423412\\ 
85	0.14596481780800996\\ 
86	0.1427306125466207\\ 
94	0.13922672942973757\\ 
103	0.1355129881697659\\ 
119	0.13122549068082925\\ 
133	0.12654465994363093\\ 
134	0.12172935915284798\\ 
148	0.11639316694372408\\ 
153	0.11076574590015122\\ 
165	0.10485858198317236\\ 
182	0.09842922945015294\\ 
209	0.09094646925329392\\ 
233	0.08234505515801471\\ 
245	0.07350532277259954\\ 
248	0.06430112139051922\\ 
299	0.05383113499046196\\ 
701	0.028731797412457714\\ 
};
\addlegendentry{Random friend degree distribution}

	\end{axis}
\end{tikzpicture}

		\caption{degree\_distribution tau2,5 loglog.tikz}
	\end{figure}

	\begin{figure}[!htb]
		\centering
		\begin{tikzpicture}
\begin{axis}[%
ybar interval,
axis x line*=bottom,
axis y line*=left,
axis background/.style={fill=white},
xmin=1, 
xmax=9,
ymax=0.45,
ymin=0,
grid=none,
ylabel={Proportion},
xlabel={Typical Distance}]
\addplot[fill=mycolor1] coordinates { 
    (1,	0.0001) 
    (2,	0.00039868)
    (3,	0.01932502) 
    (4,	0.1522872) 
    (5,	0.43066926) 
    (6,	0.33955322) 
    (7,	0.05640086) 
    (8,	0.00126376)
    (9,	2e-06)
 };
\end{axis}
\end{tikzpicture}
		\caption{typical tau2,5 distance}
	\end{figure}

	\begin{figure}[!htb]
		\centering
		%Degree distributions of PAM (t=2000, delta=1, m=2), follow powerlaw (tau=3.5) by RGCN volume 2, 8.4.11 en 8.4.12. 
\tikzstyle{every node}=[font=\normalsize]

\begin{tikzpicture}
    \begin{axis}[%
width=0.951\fwidth,
height=\fheight,
at={(0\fwidth,0\fheight)},
scale only axis,
xminorticks=true,
yminorticks=true,
axis x line*=bottom,
axis y line*=left,
x tick label style = {/pgf/number format/.cd, scaled x ticks = false, set thousands separator={}, fixed},
axis background/.style={fill=white},
legend style={legend pos=north east, legend cell align=left, align=left, draw=white!15!black}, 
xmin=0, 
xmax=130, 
ymin=0, 
ymax=1, 
xlabel={Degrees}, 
ylabel={$P(X > x)$}]

\addplot [color=mycolor2, line width = 2pt, mark size=1.5pt, mark=*, only marks, mark options={solid, mycolor2}]
table[row sep=crcr]{
1	0.9985\\ 
2	0.5264999999999997\\ 
3	0.3249999999999996\\ 
4	0.20599999999999952\\ 
5	0.14999999999999947\\ 
6	0.11549999999999944\\ 
7	0.09649999999999942\\ 
8	0.0729999999999994\\ 
9	0.058499999999999386\\ 
10	0.04949999999999938\\ 
11	0.03749999999999937\\ 
12	0.03299999999999936\\ 
13	0.028999999999999363\\ 
14	0.025999999999999364\\ 
15	0.020499999999999362\\ 
16	0.017499999999999363\\ 
17	0.015999999999999362\\ 
18	0.014999999999999361\\ 
19	0.013499999999999361\\ 
20	0.011499999999999361\\ 
21	0.009999999999999362\\ 
22	0.008999999999999363\\ 
23	0.006999999999999363\\ 
25	0.005999999999999363\\ 
26	0.005499999999999363\\ 
27	0.004999999999999363\\ 
29	0.004499999999999362\\ 
31	0.003999999999999362\\ 
34	0.0029999999999993617\\ 
42	0.0024999999999993617\\ 
45	0.0014999999999993617\\ 
48	0.0009999999999993616\\ 
80	0.0004999999999993616\\ 
130	-6.38378239159465e-16\\ 
};
\addlegendentry{Degree distribution}

\addplot [color=mycolor1, line width = 2pt, mark size=2pt, mark=diamond*, only marks, mark options={solid, mycolor1}]
table[row sep=crcr]{
1	0.9996237772761475\\ 
2	0.7628542763982944\\ 
3	0.6112365186857286\\ 
4	0.491848507649862\\ 
5	0.4216202658640581\\ 
6	0.36970152997241024\\ 
7	0.3363431151241534\\ 
8	0.2891898670679708\\ 
9	0.25645849009280147\\ 
10	0.23388512666165023\\ 
11	0.2007775269626284\\ 
12	0.18723350890393767\\ 
13	0.17419112114371696\\ 
14	0.1636568848758464\\ 
15	0.14296463506395776\\ 
16	0.1309255079006771\\ 
17	0.12452972159518427\\ 
18	0.12001504890895402\\ 
19	0.11286681715575614\\ 
20	0.10283421118635559\\ 
21	0.09493353398545266\\ 
22	0.08941560070228237\\ 
23	0.07787810383747175\\ 
25	0.07160772510659641\\ 
26	0.06834712816654123\\ 
27	0.06496112365186855\\ 
29	0.061324303987960856\\ 
31	0.05743666917481814\\ 
34	0.04890895410082768\\ 
42	0.04364183596689239\\ 
45	0.03235515425131678\\ 
48	0.026335590669676452\\ 
80	0.016302984700275906\\ 
130	2.0816681711721685e-17\\ 
};
\addlegendentry{Size biased degree distribution}

\addplot [color=mycolor3, line width = 2pt, mark size=1.5pt, mark=square*, only marks, mark options={solid, mycolor3}]
table[row sep=crcr]{
1	0.9998078063241107\\ 
2	0.8459293525602285\\ 
3	0.701272208608933\\ 
4	0.5734240952495504\\ 
5	0.49506309470763815\\ 
6	0.43294015845329537\\ 
7	0.3942591823312087\\ 
8	0.33926123581900935\\ 
9	0.2978139395715988\\ 
10	0.2705817111566972\\ 
11	0.23136191500299452\\ 
12	0.21678052921037472\\ 
13	0.20231603531564105\\ 
14	0.1910319028764428\\ 
15	0.1664902569751251\\ 
16	0.15154180035927717\\ 
17	0.1426176336926105\\ 
18	0.1368584428834197\\ 
19	0.12799178865850086\\ 
20	0.11595651053792862\\ 
21	0.10595880110941423\\ 
22	0.09989775268071875\\ 
23	0.08684160623702077\\ 
25	0.07998383824520866\\ 
26	0.07685889166401208\\ 
27	0.073019203574324\\ 
29	0.06880763855340542\\ 
31	0.06415928877871294\\ 
34	0.05412416901496414\\ 
42	0.04843076764977585\\ 
45	0.035645726066607666\\ 
48	0.029780731270270158\\ 
80	0.018867535924482107\\ 
130	0.00016673427417985415\\ 
};
\addlegendentry{Random friend degree distribution}

	\end{axis}
\end{tikzpicture}

		\caption{Degree distribution tau3,5}
	\end{figure}
	
	\begin{figure}[!htb]
		\centering
		%Loglog Degree distributions of PAM (t=10000, delta=1, m=2), follow powerlaw (tau=3.5) by RGCN volume 2, 8.4.11 en 8.4.12. 
\tikzstyle{every node}=[font=\normalsize]

\begin{tikzpicture}
    \begin{axis}[%
width=0.951\fwidth,
height=\fheight,
at={(0\fwidth,0\fheight)},
scale only axis,
xminorticks=true,
yminorticks=true,
axis x line*=bottom,
axis y line*=left,
x tick label style = {/pgf/number format/.cd, scaled x ticks = false, set thousands separator={}, fixed},
axis background/.style={fill=white},
legend style={legend pos=south west, legend cell align=left, align=left, draw=white!15!black}, 
xmin=0, 
xmax=153, 
ymin=0, 
ymax=1, 
xmode=log, 
ymode=log, 
xlabel={Degrees}, 
ylabel={$P(X > x)$}]

\addplot [color=mycolor2, line width = 2pt, mark size=1.5pt, mark=*, only marks, mark options={solid, mycolor2}]
table[row sep=crcr]{
1	0.9998\\ 
2	0.543500000000034\\ 
3	0.3375000000000403\\ 
4	0.2242000000000381\\ 
5	0.1581000000000372\\ 
6	0.11720000000003707\\ 
7	0.08860000000003718\\ 
8	0.06820000000003724\\ 
9	0.05610000000003726\\ 
10	0.04420000000003728\\ 
11	0.037300000000037276\\ 
12	0.031700000000037275\\ 
13	0.026200000000037273\\ 
14	0.021100000000037273\\ 
15	0.01920000000003727\\ 
16	0.01700000000003727\\ 
17	0.01510000000003727\\ 
18	0.013300000000037268\\ 
19	0.011500000000037267\\ 
20	0.009700000000037266\\ 
21	0.008900000000037265\\ 
22	0.008400000000037265\\ 
23	0.008100000000037265\\ 
24	0.007400000000037265\\ 
25	0.007000000000037265\\ 
26	0.006500000000037264\\ 
27	0.006200000000037264\\ 
28	0.005700000000037264\\ 
29	0.0049000000000372634\\ 
30	0.0046000000000372635\\ 
31	0.004300000000037264\\ 
32	0.0038000000000372636\\ 
33	0.0036000000000372635\\ 
34	0.0034000000000372634\\ 
35	0.0031000000000372635\\ 
36	0.0030000000000372637\\ 
37	0.0026000000000372635\\ 
39	0.0022000000000372633\\ 
40	0.0019000000000372634\\ 
41	0.0018000000000372633\\ 
42	0.0017000000000372633\\ 
43	0.0015000000000372632\\ 
44	0.001300000000037263\\ 
46	0.001100000000037263\\ 
50	0.001000000000037263\\ 
52	0.000800000000037263\\ 
54	0.0007000000000372629\\ 
58	0.0006000000000372629\\ 
60	0.0005000000000372628\\ 
72	0.00040000000003726286\\ 
99	0.00030000000003726286\\ 
115	0.00020000000003726287\\ 
153	0.00010000000003726287\\ 
};
\addlegendentry{Degree distribution}

\addplot [color=mycolor1, line width = 2pt, mark size=2pt, mark=diamond*, only marks, mark options={solid, mycolor1}]
table[row sep=crcr]{
1	0.9999499699819892\\ 
2	0.7716629977986919\\ 
3	0.6170702421453017\\ 
4	0.5037022213328098\\ 
5	0.42102761656994947\\ 
6	0.35964078447068853\\ 
7	0.30956073644187054\\ 
8	0.26873624174505184\\ 
9	0.24149489693816723\\ 
10	0.21172703622173686\\ 
11	0.19274064438663543\\ 
12	0.17593055833500415\\ 
13	0.1580448268961405\\ 
14	0.14018411046628224\\ 
15	0.13305483289974218\\ 
16	0.12424954972984008\\ 
17	0.11616970182109468\\ 
18	0.10806483890334387\\ 
19	0.0995097058234958\\ 
20	0.09050430258155047\\ 
21	0.08630178106864264\\ 
22	0.08355013007804823\\ 
23	0.08182409445667538\\ 
24	0.07762157294376755\\ 
25	0.07512007204322718\\ 
26	0.0718681208725247\\ 
27	0.069841905143087\\ 
28	0.06633980388233048\\ 
29	0.06053632179307682\\ 
30	0.05828497098259048\\ 
31	0.05595857514508793\\ 
32	0.05195617370422334\\ 
33	0.050305183109866695\\ 
34	0.04860416249749924\\ 
35	0.045977586551931855\\ 
36	0.045077046227737325\\ 
37	0.04137482489493757\\ 
39	0.03747248349009459\\ 
40	0.034470682409446146\\ 
41	0.033445067040224596\\ 
42	0.03239443666199764\\ 
43	0.030243145887532925\\ 
44	0.0280418250950574\\ 
46	0.025740444266560258\\ 
50	0.024489693816290074\\ 
52	0.021888132879728087\\ 
54	0.020537322393436287\\ 
58	0.019086451871122873\\ 
60	0.01758555133079865\\ 
72	0.015784470682409582\\ 
99	0.013307984790874614\\ 
115	0.010431258755253187\\ 
153	0.00660396237742642\\ 
};
\addlegendentry{Size biased degree distribution}

\addplot [color=mycolor3, line width = 2pt, mark size=1.5pt, mark=square*, only marks, mark options={solid, mycolor3}]
table[row sep=crcr]{
1	0.99999896761735\\ 
2	0.8453271702393961\\ 
3	0.6955285004729513\\ 
4	0.5722595797194716\\ 
5	0.47965061823272037\\ 
6	0.4086411143892153\\ 
7	0.3495939225247997\\ 
8	0.30169790384224193\\ 
9	0.27105444303774984\\ 
10	0.23727642826582393\\ 
11	0.21570694026542284\\ 
12	0.19564928425852227\\ 
13	0.1749143573261505\\ 
14	0.15408876566347912\\ 
15	0.1459460517682565\\ 
16	0.13597109265013943\\ 
17	0.12681494478313582\\ 
18	0.11752363864836674\\ 
19	0.10805641768109683\\ 
20	0.09770642597082155\\ 
21	0.09343192801765346\\ 
22	0.09051419660964662\\ 
23	0.08869578673257295\\ 
24	0.083732133220724\\ 
25	0.08109281881959962\\ 
26	0.07713998450407568\\ 
27	0.07503482659547252\\ 
28	0.07131781876965482\\ 
29	0.0645620470990338\\ 
30	0.06213550497260935\\ 
31	0.05932588442682781\\ 
32	0.054929436943443447\\ 
33	0.05339462820272148\\ 
34	0.05175860345909593\\ 
35	0.0492563580202874\\ 
36	0.04839434376642532\\ 
37	0.04447150262763347\\ 
39	0.04071296555326248\\ 
40	0.0369828462505547\\ 
41	0.035762573608164414\\ 
42	0.03480509225188769\\ 
43	0.03220239360056481\\ 
44	0.029622914277335384\\ 
46	0.027119062315945193\\ 
50	0.025821604870525684\\ 
52	0.022832716597353264\\ 
54	0.021484784881737724\\ 
58	0.01995944534246436\\ 
60	0.018240803970764167\\ 
72	0.01642424889354607\\ 
99	0.013793686152935376\\ 
115	0.01094606954668247\\ 
153	0.006855766151976607\\ 
};
\addlegendentry{Random friend degree distribution}

	\end{axis}
\end{tikzpicture}

		\caption{degree\_distribution tau3,5 loglog.tikz}
	\end{figure}
	
	\begin{figure}[!htb]
		\centering
		\begin{tikzpicture}
\begin{axis}[%
ybar interval,
axis x line*=bottom,
axis y line*=left,
axis background/.style={fill=white},
xmin=1, 
xmax=10,
ymax=0.4,
ymin=0,
grid=none,
ylabel={Proportion},
xlabel={Typical Distance}]
\addplot[fill=mycolor1] coordinates { 
    (1,	0.0001) 
    (2,	0.00039958) 
    (3,	0.0038188) 
    (4,	0.02972338) 
    (5,	0.14733288) 
    (6,	0.36466948) 
    (7,	0.35127398) 
    (8,	0.09745174) 
    (9,	0.00520142) 
    (10, 2.874e-05) 
 };
\end{axis}
\end{tikzpicture}
		\caption{typical tau3,5 distance}
	\end{figure}

\end{document}